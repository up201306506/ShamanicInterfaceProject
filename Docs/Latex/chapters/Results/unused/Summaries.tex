\begin{dummied}
\section{Placeholder Summary of Conclusions} \label{sec:summaries}
    NUI's, one of the great fields of technology in the modern tech sector, are the next step in the evolution of interfaces, expected to replace the GUI's within certain old and new use cases. But their definition is challenged with by the current state of design and technology. The central contribution that the Shamanic Interface attempts to tackle has to do with this subject. The proposal states that it would benefit gesture recognition interfaces to make a concern-separation layer between gestures performed by users and the commands the system can execute. The SI would contribute by improving natural usability and learning, and by improving the immersiveness of those systems.\\
    A previous thesis was worked on from which a research tool for testing a SI implementation was developed. The tool itself was tested, but there was no conclusive evaluation or empirical analysis of the SI concept itself, leaving that as future tasks for the present work. Some improvements and alterations to the tool itself may be required, and as such, some of the concepts behind its inception were looked at. One such change is the need to include more meaningfully rich gestures from cultures among the commands the virtual environment’s tasks demands.\\ 
    The essential aim of the thesis will be to perform field studies with the modified tool. Field research are data gathering activities performed in a natural context. In this case, the intent is to perform direct observation of user behaviour when attempting to complete tasks with the tool, with small involvement of the observer. To meet the goals, participants should be of at least two separate cultural backgrounds and they will need to perform on two sessions set apart some time from each other.\\
\end{dummied}