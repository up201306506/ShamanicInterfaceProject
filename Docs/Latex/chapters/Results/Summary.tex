\section{Conclusions} \label{sec:results_Conclusions}
    This work reported on a number of findings of two sets of user trials to assess impacts of measures on the quality of several aspects of UX, by means of, primarily, a comparative test. The purpose of these findings was to validate three initial propositions offered before the start of planning and development of the work. Of these three items, two have been found conclusive, while another did not come to be, and should require a different approach with a number of alterations.\\
    \begin{itemize}
        \item \textbf{Focusing on user culture contributes to the learning rate and capacity of commands; and to the Retention and Memorization of commands and concepts}\\
            On several passages throughout section \ref{sec:results_game} it is referenced that the majority of the Cultural Group performed exceptionally well during both game trials. Very few errors and failures were registered on the first trial, and coming into the second, the group even flourishingly carried on to manifest growth and improvement over the earlier trial. This in contrast to the Non-Cultral Group, which had trouble assimilating and present the corect commands, and in due, saw a worsening development of performance after the two week delay. Aspects of confidence, anticipation and memory were also among the descriptors looked into that illustrated a reduced positive impact on the Non-Cultural Group independent from the performance evaluation. Coming back to said evaluation, it becomes palpable that the Non-Cultural Group didn't just become unsucessful and fell short based on the choice of gesture set given, but moreover, also due to the group seeking out the what was the approapriate gesture for the game's context, which coincided with the Cultural Group's gesture set. This persistence even poisons the Non-Cultural recolection, making them feel certainty and assurance which has been ultimately misplaced, indicating that the Cultural gesture set was not just prefered among the two, but potentially, the most natural choice for the situation.
            
        \item \textbf{Focusing on user culture contributes to the satisfaction and immersion of the experience}\\
            Two measures were performed primarily towards this during the work. The usability report was insightful, proving that there was a highly significant difference \emph{(p=0.0031)} in the game between the Groups in terms of the system's capabilities to deliver a satisfying entertaining experiece. But one set of measures wasn't enough, and one another more comprehensive invstigation into immersiveness and satisfaction did not produce further corroboration. It's not improbable that this benefit is still factual, and it's possible to suppose from certain observations and moments in the game, particularly involving task T3, that this could be better observed with greater quality of superficial design.
    \end{itemize}

\section{Closing Thoughts and Future Work} \label{sec:results_End}
    With this work, we have consolidated the idea that cultural emblems are natural and can symbolically benefit users to organically with the initial barrier of entry to using a NUI. By testing a game built around the context of solving gestures, users are prone to resorting to these emblems they know of, and are moreover, it makes it accessible if recall the commands if necessary. But more needs to be tested and done so more rigorously.\\
    Satisfaction and Immersion have not been properly justified, as noted above, but there's more to investigate. The current findings all aim towards benefits involving the initial usage hurdle of an application, it tests out easy to perform tasks, even if these involve multiple steps. While there was a plan to look into separating tasks into easy and hard tasks, this never quite came into fruition. The justification for Optional and Mandatory tasks was such that, besides the very first optional task, these would all involve larger struggle towards resolution. Some ideas for what this difficulty would ensue were written down, such involving gesturing towards objects with motions that would make sense only if aiming to address an intelligent receiver. However, neither was any solution really acceptable, the approach itself didn't seem very lenient towards a good analysis. Still, it would be deemed really useful to make one potential analysis of the upper parameters of complexity. To test out if adaptable Cultural Gestures do provide a benefit to Depth of Interaction. While learning and memory as advantages may be upheld right now, what degree of richness the interaction achieves is not.\\
    Another aspect of the application was unsatisfactory. Player Movement was too contrived by a composite issue of the transition between movement gestures, the Leap Motion having certain faults in perceiving movement with the palm pointed upwards, and with the game itself not smoothing out the rotation speed of turning very gracefully. However, in hindsight, it's very possible that movement in general may have not been a very insightful act to observe in the first place. Additionally, movement is not something that is customarily performed in a natural fashion even in recent immersive systems, such as virtual reality experiences, which relies heavily on either unconventional player teleportation to achieve large distances or using a controller’s analog stick for advancing through the world, both approaches which clearly unfit for a fully Natural Interface.\\
    \\
    Thus we suggest the following directives towards a development of future system: Improve the Classifier, bring a new and better approach to the recognition of hand gestures such as machine learning approaches; Create a Virtual Reality environment to test the participants in a more immersive setting, or alternatively, create an Amplified Reality game such that the environment is virtually super-imposed over the real world setting; Focus primarily on the Task setup, remove long duration requirements between each focal point of the observation, such as movement; Make a good definition of task failure and give players a fixed number of attempts or time to complete them, make sure to employ proper feedback within the application on both aspects; Create a difficulty curve, make it so some of the tasks are clearly devised as more difficult than others; Test multiple cultural settings, find valid gesture sets for more than one culture and verify that the benefits permeate all of these over a non-cultural control group.\\
    