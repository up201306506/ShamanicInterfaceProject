\section{Cultural Experience Validation} \label{sec:results_culturalvalidation}
    The pre-test was a common section that all 17 volunteers performed, still unaware of the User Trial’s goals or even general setup beyond the knowledge that cultural gestures would be an essential part of the observation. With no pre-emptive knowledge of the experience and with, at this point, very little suggestive information as to what will be required of them, they were surveyed, using simple language and a simple presentation, on what they found was the correct set of emblematic gestures that would solve or address each of the specific Tasks they would later have to employ in the game. Before beginning, the users were assured that no such thing as a wrong answer could possibly exist as this was a direct survey of their personal opinions, and they were put at ease that none of the data lifted would result into later questioning, as this data would have no relation with the experiment ahead.\\
    Ideally, this would be performed on a different set of users beforehand, which would even allow for a restructuring of the methodology employed on the user trials as a whole, exploring aspects such as natural discovery of commands without guidance. However, given the limited availability of volunteers and the likelihood of dropout in a two-part experiment, it was found hard to do so within the means reachable.\\
    If a Task’s chosen gesture is correctly selected and fit for the purposes of a Cultural Experience, then when given the opportunity to perform the Task on any independent context, a person should most of the time opt to perform that same well-fit gesture over another. Admittedly, certain scenarios may change the chosen gesture due to certain contextual cues lending towards different interpretations, and on top of that, not all gestures that abide by this are necessarily emblematic. For example, working with certain tools may represented by an emblematic gesture, by limiting the shape of the tool to a different arrangement may lead the person to perform a less emblematic gesture, and instead perform mimicry of the action. This concern was actually predicated in the design of the game by the optional Task O3, answering a phone, which demanded the choice on in-game representation of the ringing Phone be an object with a handset piece that can be held, as opposed, for example, a more modern looking device. This is why in the pre-test, not just were users asked to perform the gesture they find correct without a particular context in the barebones slideshow, but they were also asked if the game’s taught gesture made sense for the context.\\

\subsection{Invalidated Tasks} \label{sec:results_invalid_tasks}
    Table \ref{tab:Table_GestureValidationPretest} is the result of the first part of the pre-test, where all of the volunteers were asked about performing the chosen gesture.

    \begin{table}[ht]
    \begin{tabular}{|l|l|l|l|l|l|l|l|l|l|}
    \hline
    \cellcolor[HTML]{FFFFFF} & O1                        & T1                        & O2                        & T2                        & O3                        & T3.1                      & T3.2                      & O4                        & T4                        \\ \hline
    V1                       & \cellcolor[HTML]{D6FDD5}Y & \cellcolor[HTML]{D6FDD5}Y & \cellcolor[HTML]{FFE7E6}N & \cellcolor[HTML]{FFE7E6}N & \cellcolor[HTML]{D6FDD5}Y & \cellcolor[HTML]{FFE7E6}N & \cellcolor[HTML]{D6FDD5}Y & \cellcolor[HTML]{D6FDD5}Y & \cellcolor[HTML]{D6FDD5}Y \\ \hline
    V2                       & \cellcolor[HTML]{D6FDD5}Y & \cellcolor[HTML]{D6FDD5}Y & \cellcolor[HTML]{FFE7E6}N & \cellcolor[HTML]{D6FDD5}Y & \cellcolor[HTML]{D6FDD5}Y & \cellcolor[HTML]{D6FDD5}Y & \cellcolor[HTML]{D6FDD5}Y & \cellcolor[HTML]{D6FDD5}Y & \cellcolor[HTML]{D6FDD5}Y \\ \hline
    V3                       & \cellcolor[HTML]{D6FDD5}Y & \cellcolor[HTML]{D6FDD5}Y & \cellcolor[HTML]{D6FDD5}Y & \cellcolor[HTML]{FFE7E6}N & \cellcolor[HTML]{D6FDD5}Y & \cellcolor[HTML]{D6FDD5}Y & \cellcolor[HTML]{D6FDD5}Y & \cellcolor[HTML]{D6FDD5}Y & \cellcolor[HTML]{FFE7E6}N \\ \hline
    V4                       & \cellcolor[HTML]{FFE7E6}N & \cellcolor[HTML]{D6FDD5}Y & \cellcolor[HTML]{FFE7E6}N & \cellcolor[HTML]{D6FDD5}Y & \cellcolor[HTML]{D6FDD5}Y & \cellcolor[HTML]{D6FDD5}Y & \cellcolor[HTML]{D6FDD5}Y & \cellcolor[HTML]{D6FDD5}Y & \cellcolor[HTML]{D6FDD5}Y \\ \hline
    V5                       & \cellcolor[HTML]{D6FDD5}Y & \cellcolor[HTML]{D6FDD5}Y & \cellcolor[HTML]{FFE7E6}N & \cellcolor[HTML]{D6FDD5}Y & \cellcolor[HTML]{D6FDD5}Y & \cellcolor[HTML]{D6FDD5}Y & \cellcolor[HTML]{D6FDD5}Y & \cellcolor[HTML]{D6FDD5}Y & \cellcolor[HTML]{D6FDD5}Y \\ \hline
    V6                       & \cellcolor[HTML]{D6FDD5}Y & \cellcolor[HTML]{D6FDD5}Y & \cellcolor[HTML]{FFE7E6}N & \cellcolor[HTML]{FFE7E6}N & \cellcolor[HTML]{D6FDD5}Y & \cellcolor[HTML]{FFE7E6}N & \cellcolor[HTML]{D6FDD5}Y & \cellcolor[HTML]{D6FDD5}Y & \cellcolor[HTML]{D6FDD5}Y \\ \hline
    V7                       & \cellcolor[HTML]{FFE7E6}N & \cellcolor[HTML]{D6FDD5}Y & \cellcolor[HTML]{FFE7E6}N & \cellcolor[HTML]{FFE7E6}N & \cellcolor[HTML]{D6FDD5}Y & \cellcolor[HTML]{D6FDD5}Y & \cellcolor[HTML]{D6FDD5}Y & \cellcolor[HTML]{D6FDD5}Y & \cellcolor[HTML]{D6FDD5}Y \\ \hline
    V8                       & \cellcolor[HTML]{D6FDD5}Y & \cellcolor[HTML]{D6FDD5}Y & \cellcolor[HTML]{FFE7E6}N & \cellcolor[HTML]{FFE7E6}N & \cellcolor[HTML]{D6FDD5}Y & \cellcolor[HTML]{D6FDD5}Y & \cellcolor[HTML]{D6FDD5}Y & \cellcolor[HTML]{D6FDD5}Y & \cellcolor[HTML]{D6FDD5}Y \\ \hline
    V9                       & \cellcolor[HTML]{D6FDD5}Y & \cellcolor[HTML]{D6FDD5}Y & \cellcolor[HTML]{FFE7E6}N & \cellcolor[HTML]{FFE7E6}N & \cellcolor[HTML]{D6FDD5}Y & \cellcolor[HTML]{FFE7E6}N & \cellcolor[HTML]{D6FDD5}Y & \cellcolor[HTML]{D6FDD5}Y & \cellcolor[HTML]{D6FDD5}Y \\ \hline
    V10                      & \cellcolor[HTML]{D6FDD5}Y & \cellcolor[HTML]{D6FDD5}Y & \cellcolor[HTML]{FFE7E6}N & \cellcolor[HTML]{FFE7E6}N & \cellcolor[HTML]{D6FDD5}Y & \cellcolor[HTML]{FFE7E6}N & \cellcolor[HTML]{D6FDD5}Y & \cellcolor[HTML]{D6FDD5}Y & \cellcolor[HTML]{D6FDD5}Y \\ \hline
    V11                      & \cellcolor[HTML]{D6FDD5}Y & \cellcolor[HTML]{D6FDD5}Y & \cellcolor[HTML]{FFE7E6}N & \cellcolor[HTML]{FFE7E6}N & \cellcolor[HTML]{D6FDD5}Y & \cellcolor[HTML]{D6FDD5}Y & \cellcolor[HTML]{D6FDD5}Y & \cellcolor[HTML]{D6FDD5}Y & \cellcolor[HTML]{D6FDD5}Y \\ \hline
    V12                      & \cellcolor[HTML]{D6FDD5}Y & \cellcolor[HTML]{D6FDD5}Y & \cellcolor[HTML]{FFE7E6}N & \cellcolor[HTML]{FFE7E6}N & \cellcolor[HTML]{D6FDD5}Y & \cellcolor[HTML]{FFE7E6}N & \cellcolor[HTML]{D6FDD5}Y & \cellcolor[HTML]{D6FDD5}Y & \cellcolor[HTML]{D6FDD5}Y \\ \hline
    V13                      & \cellcolor[HTML]{D6FDD5}Y & \cellcolor[HTML]{D6FDD5}Y & \cellcolor[HTML]{FFE7E6}N & \cellcolor[HTML]{FFE7E6}N & \cellcolor[HTML]{D6FDD5}Y & \cellcolor[HTML]{D6FDD5}Y & \cellcolor[HTML]{D6FDD5}Y & \cellcolor[HTML]{FFE7E6}N & \cellcolor[HTML]{D6FDD5}Y \\ \hline
    V14                      & \cellcolor[HTML]{D6FDD5}Y & \cellcolor[HTML]{D6FDD5}Y & \cellcolor[HTML]{FFE7E6}N & \cellcolor[HTML]{FFE7E6}N & \cellcolor[HTML]{D6FDD5}Y & \cellcolor[HTML]{D6FDD5}Y & \cellcolor[HTML]{D6FDD5}Y & \cellcolor[HTML]{D6FDD5}Y & \cellcolor[HTML]{D6FDD5}Y \\ \hline
    V15                      & \cellcolor[HTML]{D6FDD5}Y & \cellcolor[HTML]{D6FDD5}Y & \cellcolor[HTML]{FFE7E6}N & \cellcolor[HTML]{FFE7E6}N & \cellcolor[HTML]{FFE7E6}N & \cellcolor[HTML]{FFE7E6}N & \cellcolor[HTML]{D6FDD5}Y & \cellcolor[HTML]{D6FDD5}Y & \cellcolor[HTML]{D6FDD5}Y \\ \hline
    V16                      & \cellcolor[HTML]{D6FDD5}Y & \cellcolor[HTML]{D6FDD5}Y & \cellcolor[HTML]{FFE7E6}N & \cellcolor[HTML]{FFE7E6}N & \cellcolor[HTML]{D6FDD5}Y & \cellcolor[HTML]{D6FDD5}Y & \cellcolor[HTML]{D6FDD5}Y & \cellcolor[HTML]{D6FDD5}Y & \cellcolor[HTML]{D6FDD5}Y \\ \hline
    V17                      & \cellcolor[HTML]{D6FDD5}Y & \cellcolor[HTML]{D6FDD5}Y & \cellcolor[HTML]{FFE7E6}N & \cellcolor[HTML]{FFE7E6}N & \cellcolor[HTML]{D6FDD5}Y & \cellcolor[HTML]{D6FDD5}Y & \cellcolor[HTML]{D6FDD5}Y & \cellcolor[HTML]{D6FDD5}Y & \cellcolor[HTML]{FFE7E6}N \\ \hline
    \end{tabular}
    \caption{\label{tab:Table_GestureValidationPretest}User-oriented Validation of Task gesture choice by mean of blind survey of natural cultural emblems.}
    \end{table}
    
    Two tasks with concerning response arise in a suspiciously distinct manner. Of the one or two gestures each user made for the slide correspondent to Task O2 and Task T2, only very few actually performed the expected Gesture during the Cultural Validation. This punctuates the possibility of having made the wrong assumption for each of the two during the initial brainstorming or subsequent steps involved in this selection of Gestures. Alternatively, it could also signify that the presentation itself was misleading or not clear enough.\\
    To that end, all the diverging answers provided by the users were recorded, in an attempt to figure out if there’s a consistent pattern as to what went wrong in the design of the Task.\\
    Task O2 involves taking a snapshot of a flower and framing it in an initially empty rectangular mount. Already, with care, from this Task description could the problem be perceived, as well as the eventual source of the disparity. The Task describes two actions, the photograph snapping, and framing of the revealed photo. While in design, the action focused upon was the framing action, most users instead focused on the photographic action itself, which involved a mimicry of the snapshot acting. Out of all 16 users that performed a different action, every single one was recorded performing a gesture called the \emph{“Camera Click”}. This gesture was not found as a symbolic emblem while first researching and planning for this work, which was partially the reason that led to the wrong assumption that it wouldn’t be the most natural response to the task’s proceedings. There was some worry about this Task towards the impact it may have had on the global experience, however this worry was put at ease with the second component of this pre-test, as can be seen on table \ref{tab:Table_GestureVerification}, every user in the cultural group stated that the \emph{Framing} emblem was recognizable and fit the game’s context. Every user, but one, Volunteer number 11, who left the comment “Knows the gesture, but wouldn’t see the connection without an explanation”.\\
    Task T2 is a slightly more complicated matter. The task features a large clock and a waiting sign room, and the users are meant to emote their impatience at a long wait. Only three people gave the expected gesture of looking at or tapping the opposite wrist, however, when looking over the other given gestures, no pattern was recognized. A group of three volunteers performed another well fit gesture of bringing their hand to the side of their head, while two volunteers crossed their arms and tried appearing displeased, a gesture that doesn’t involve detectable hand movements. The remaining gestures were singleton examples of emblems corresponding to hurrying other people, threats and then other non-emblematic gestures of disconcerting reasoning. Unlike with the O2 task, were all users performed one, this makes a total of only 7 users performing an adequate emblematic gesture for the given scenario in T2, and 5 users performing non-emblematic gestures. This is an indication that the scenario chosen for task T2 may have not been an apt one for this game.\\
    As such, the data collected for these two tasks is ignored while doing a discrete assessment of the Tasks, since it can’t be fully justified that differences found between the groups are due to the cultural differences of the experience. These, however, still take a part of the global assessment of the experience, as table \ref{tab:Table_GestureVerification} does provide verification that the users understand and recognize that the choice of gesture is fit to both Tasks, even if not their first choice.\\



\subsection{Preeminent Non-Cultural Misfits} \label{sec:results_pretest_noncultural}
    Past the Cultural Validation, volunteers were taught the game’s intended gestures and asked to comment on them, particularly by request of mentioning if they had prior knowledge of the emblem required and if they found it fitting to the Task shown. Table \ref{tab:Table_GestureVerification} is the result of this request. As is evident, the Cultural group almost completely shown agreement with the choice of gesture. This is was an expected outcome.\\
    But for the Non-Cultural Group, the intent was never to obtain an acquiescent response, but rather a mixture. The objective of the gesture selection for the Non-Cultural Group was to avoid the natural choice, but still pick something that could make sense given an alternate context of the task. For example, the task O1, silencing an object, is performed by showing a thumbs down gesture of disapproval at the speaker prop. The contrast between what’s asked and what gesture is required at play here is that, while the thumbs down is an emblematic gesture for the volunteers and could be used in a different scenario where the objective is communicating the request of lowering of a music’s volume, this is not very sensical if the recipient of the gesture isn’t a person. Ideally, all tasks would have a balanced mixture of volunteers finding the gesture fitting and unfitting, but some gestures have got an overtly positive or overtly negative response.\\
    The Non-Cultural Group participants seemed to focus most on the negative at this step, which is why the presence of task O2 as a major negative read is preeminent. For the positively approached gestures, a difference between the response and the actual performance in practice, specially on the Second Trial, could provide more insight than just user opinion. But for a preeminent negative task, this means that the gesture chosen for the non-cultural group may be far too detached from the task to make a fair comparison between it and a cultural equivalent. The reasoning is that this unique contrast between O2 and the rest of tasks may make it more memorable or notable than the remainder for the Non-Cultural, which is an undesirable factor for the results obtained.\\
    As such, this compounds that Task O2 should be ignored during data analysis. Users were not informed and were still requested to perform it if they felt like it alongside the remaining Optional Tasks, so as to not alter the experience between first and second trial. No special attention being brought to it or to Task T2.\\


    \begin{table}[ht]
    \begin{tabular}{|l|l|l|l|l|l|l|l|l|l|}
    \hline
    Cultural Group & O1 & T1 & O2             & T2 & O3 & T3.1 & T3.2 & O4 & T4 \\ \hline
    V1  & Y  & Y  & Y                         & Y  & Y  & Y    & Y    & Y  & Y  \\ \hline
    V3  & Y  & Y  & Y                         & Y  & Y  & Y    & Y    & Y  & Y  \\ \hline
    V5  & Y  & Y  & Y                         & Y  & Y  & Y    & Y    & Y  & Y  \\ \hline
    V6  & Y  & Y  & Y                         & Y  & Y  & Y    & Y    & Y  & Y  \\ \hline
    V9  & Y  & Y  & Y                         & Y  & Y  & Y    & Y    & Y  & Y  \\ \hline
    V11 & Y  & Y  & \cellcolor[HTML]{FFE7E6}N & Y  & Y  & Y    & Y    & Y  & Y  \\ \hline
    V13 & Y  & Y  & Y                         & Y  & Y  & Y    & Y    & Y  & Y  \\ \hline
    V15 & Y  & Y  & Y                         & Y  & Y  & Y    & Y    & Y  & Y  \\ \hline
    V17 & Y  & Y  & Y                         & Y  & Y  & Y    & Y    & Y  & Y  \\ \hline
    \end{tabular}
    \newline
    \newline
    \newline
    \begin{tabular}{|l|l|l|l|l|l|l|l|l|l|}
    \hline
    Non-Cultural Group & O1 & T1 & O2 & T2 & O3 & T3.1 & T3.2 & O4 & T4 \\ \hline
    V2 & Y & Y & \cellcolor[HTML]{FFE7E6}N & Y & Y & Y & Y & Y & Y \\ \hline
    V4 & \cellcolor[HTML]{FFE7E6}N & Y & Y & Y & \cellcolor[HTML]{FFE7E6}N & Y & Y & Y & Y \\ \hline
    V7 & Y & \cellcolor[HTML]{FFE7E6}N & \cellcolor[HTML]{FFE7E6}N & Y & Y & Y & \cellcolor[HTML]{FFE7E6}N & Y & Y \\ \hline
    V8 & \cellcolor[HTML]{FFE7E6}N & Y & \cellcolor[HTML]{FFE7E6}N & Y & Y & Y & Y & Y & Y \\ \hline
    V10 & Y & Y & \cellcolor[HTML]{FFE7E6}N & Y & \cellcolor[HTML]{FFE7E6}N & Y & Y & Y & Y \\ \hline
    V12 & Y & \cellcolor[HTML]{FFE7E6}N & \cellcolor[HTML]{FFE7E6}N & Y & Y & \cellcolor[HTML]{FFE7E6}N & \cellcolor[HTML]{FFE7E6}N & Y & Y \\ \hline
    V14 & Y & Y & \cellcolor[HTML]{FFE7E6}N & Y & \cellcolor[HTML]{FFE7E6}N & Y & Y & Y & Y \\ \hline
    V16 & Y & Y & \cellcolor[HTML]{FFE7E6}N & Y & Y & Y & Y & \cellcolor[HTML]{FFE7E6}N & Y \\ \hline
    \end{tabular}
    \caption{\label{tab:Table_GestureVerification}Difference in consensus and recognition between gestures used in the Cultural and Non-Cultural Group.}
    \end{table}


