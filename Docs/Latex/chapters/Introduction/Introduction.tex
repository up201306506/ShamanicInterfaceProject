\chapter{Introduction} \label{chap:intro}
\section{Context} \label{sec:intro_context}
    %Esta secção descreve a área em que o trabalho se insere, podendo referir um eventual projeto de que faz parte e apresentar uma breve descrição da empresa onde o trabalho decorreu.
    %Presença e importância dada a interfaces naturais, aparelhos e aplicações habituais em que se encontram. Detecção de Gestos e como se aplicam à área, potencial de crescente importância
    Body Motion and Gesture recognition has seen an interesting development with modern technology. We already now see it employed with many forms of sensors and applications. The most common of which includes the touchscreens found on smart devices, which became one of the basic innovations that would shape up the future of interaction between end-user and machine, as well as user expectations for current tech.\\
    Any kind of movement performed by a human can be classified as a gesture, but not all gestures can be considered ‘natural’. Natural, in this context, is assumed to be a set of behaviours that come effortlessly and intuitively to the users, in a way they may not even consciously acknowledge. This is an idea that was taken in account by Steve Mann when he introduced the Natural User Interface concept \cite{Mann:2001}. A NUI is an interface built with organic experience as its primary goal, with which the user should find a higher degree of freedom to explore it without the limitations of the technology surprising the user’s anticipations or therefore damaging the ergonomics of the interaction. Thus, it focuses on human factors, the environment and senses a person relies on. Ideally, the interface itself should be effectively invisible to the user, even as they learn to perform more complex interactions with the system. The NUI designation is also later presented as an evolution of interaction paradigms as a whole, following that of the Graphical User Interface\cite{NUIgroupHome}.\\
    Surrounding the turn of the 2010’s decade, a lot of research and development was done into NUIs, particularly in the field personal computing and entertainment. Since the release of the Wii Remote and of the Xbox’s Kinect, the gaming industry had an arms race for new true-to-life interactions methods\cite{ROCCETTI2012}, meanwhile, on mobile, accelerometers, gyroscopes, proximity sensors and compasses have become the norm and implicitly expected to be a part of any model’s feature set. Besides vision, touch or accelerometer based sensors, examples of applications commonly referred to as NUI may achieve its operation through use of voice recognition, facial expression, gaze direction and biometrics including heart rate or electromyographic sensing. Many devices for each type of input have been developed, even where similar gestures are detected, as an example, the Myo Armband and Leap Motion Controller both register motion of the hands and fingers despite different approaches.\\
    Thus, through interest in NUIs, it can be said that for Gesture Interfaces, many have been attempting to create potential standards of interaction that provides larger diversity and scope of use. While NUIs are not predicted to become a predominant form for all future interaction, it's clear they're here to stay and will carve out a mainstay niche. And the same way the GUI has not replaced the Command Line Interface, but rather lowered the barriers of entry for broader use cases and audiences to more complex degrees, the NUI will also not be replacing the GUI, but rather looks to become a facilitator for the scenarios where they do make usage and learning easier.

\section{Motivation} \label{sec:intro_motivation}
    %Apresenta a motivação e enumera os objetivos do trabalho terminando com um resumo das metodologias para a prossecução dos objetivos.
    % 1 Explicar a importância de resolver o problema identificado
    % 2 Primeiro deve vir a Motivação Cientifica e só depois a motivação pessoal
    % Aspectos actuais da interacção natural e em como eles falham. Porque é que é importante melhorar a interação natural, que tecnologias e caminhos de desenvolvimento podem dai usufruir. Interface shamaica como um novo paradigma, e aspecto adaptativos a cultura de utilizador como ponto de referência\\
    However, despite given the undeniable relevance of NUIs and their ongoing research towards modern technology, it can’t also be undeniably stated that their design philosophy has been sufficiently explored. An opinion that was summarised by Don Norman \cite{NormanNUI}\cite{NormanGIS} and has since been repeatedly cited, in regards to what he and others felt was the place NUIs currently held  during the onset of its surge. He claims that Natural Interfaces are useful, but that they may currently be a misnomer. The discussion further equates their development to the early developments of the GUI, where a lot of actions would be explained through use of metaphors. For the GUI, the popular metaphor that still survives today in vernacular, was that of a work desktop with papers and filing folders strewn about, and a hand to drag and work on them. However, the metaphor was merely a learning aid, and it doesn't directly resemble the actions intended when handling a GUI. For the Natural Interface which purports to better leverage the usage of metaphors, this was not necessarily encountered.\\
    The more remarkable example of a metaphor successfully working as a NUI, but then failing the user expectations was found with a bowling game for the Nintendo Wii system. The Wii Controller is a gestural form of input with buttons on either side allowing users to mimic the motion of grabbing and swinging a bowling ball. Users are supposed to apply pressure on the buttons, and then perform a swinging motion releasing the buttons at its end, analogous to that of using a bowling ball in a real-world environment. However, when players got invested and immersed in the game, occasionally it would be verified that they would also release the controller itself at the end of the swing, throwing the controlled most likely in the direction of the game display. This would take users out of the experience, and they'd then make their plays with more careful and inhibited, yet far less natural, impetus.\\
    The reasons given for the labelled failure of Natural Interfaces is due to them not conforming to the rules, or heuristics, of interaction design that apply beyond the scope of any particular technology\cite{Shneiderman:1997}\cite{Nielsen:1990}. Specifically, existing NUIs have issues with the visibility of signifiers and thus also with discoverability of new commands. With the freedom, reliability, feedback and, as seen above, error prevention, leading to users to perform commands they’re not even conscious of, and being unaware of how to quickly correct the program after issuing an erroneous change\cite{NormanGIS}\cite{malizia2012}. This is the lead up to a lot of scenarios where users either must be taught how to perform certain commands and are told to perform mimicry of analogue movements, or conversely, of a command that represents a non-kinesthetic concept. This proves to be confusing, and then regularly users find themselves complaining, after extended periods of use, about options that they had no way of knowing existed, of commands not making particular intuitive sense for the application, or even ones that don’t seem to work in new contexts with no discernible visible cue to explain the difference. A great deal of concern is given to the need of standards, and exploring the right approach that would actually feel seamless in the hands of users, and also to the difficulty that the behavioural distinctiveness of users presents to either goal.\\
    One the tenants in HCI as a science for its usability concerns, we are told that users should not have to ‘radically change to fit in’, but rather that ‘systems should be designed to match their requirements’\cite{Preece1993}. So, one of the solutions presented tried to face the issue regarding ambiguity of input and user spontaneity, was through adaptability. If a technology wishes to allow users to interact with it as they are used to interact with the real world in everyday life, this technology must be malleable enough to each user, and the framing by which it should handle their involuntary suppositions would be their Culture. Culture is rich in gestures and expressions that hold special meaning. Even for concepts that have no physical equivalent and thus can’t be simply produced through mimicry of the concept, the depth is such that non-verbal communication is possible. Culture aware systems could provide an answer on not just what is a valid definition of Natural for one user, but separate answers for every group of every ethnicity and upbringing.\\
    As such, there’s a need to produce research on this potential of leveraging user culture. Should a methodology prove itself to be feasible, new standards of interaction may be built for it among niches fulfilled by NUI based systems, providing users with more inclusive and immersive experiences, and opening new fields of research. The Shamanic Interface is one among ideas for introducing cultural awareness into systems, focusing on the separation of concerns between gestures classifications and virtual instructions.

\section{Goals} \label{sec:intro_goals}
    %Na continuação da secção anterior, e apenas no caso de ser um Projeto e não uma Dissertação, esta secção apresenta resumidamente o projeto.
    %Demonstrar a viabilidade e potencialidades de melhoria de interação de um utilizador com um ambiente, quando este se adaptar à condição e ao "background?" do utilizador
    The main goal for this dissertation is to explore the viability of Shamanic Interfaces as a concept and its application in interaction with virtual environments. Prior work\cite{pinto2015}, was already accomplished in building a research tool for performing field studies in a controlled environment, as well as verify its playability among a group of users from differing backgrounds. However, the empirical tests still need to be fulfilled, and the tools can still be improved. Some additional concerns were set as future work.\\
    
    Here, we seek to find empirical insights to the following encompassing hypotheses:
    
    \begin{itemize}
        \item \textbf{Focusing on user culture contributes to the Learning Rate and Capacity of commands}\\
            One of the very first aspects touched upon the Shamanic Interface proposal was a perceptible learning curve negatively impacting the user experience of Natural Interfaces featuring commands that scoped beyond of simple mimicry, or complex breadth of instructions and information. The approach to cultural richness is expectable to improve this situation and allow users more freedom thanks to culture having pre-loaded non-kinesthetic abstractions that may be used by designers, as well as permitting the application of semiotics through previously unexplored non-textual signifiers.
        
        \item \textbf{Focusing on user culture contributes to the Retention and Memorization of commands and concepts}\\
            Moreover, these benefits and observations are anticipated to produce continuous benefit over the non-cultural uses of natural interfaces, as the interaction would have a longer lasting effect on the user's affectivity, and thus, benefit them in both recollection of content, and recall based on context.
            
        \item \textbf{Focusing on user culture contributes to the Satisfaction and Immersion of the experience}\\
            Another concern sat with the receptiveness of users towards a potentially new form of interaction, and how well accepted the naturalism of human gesture when communicating with a machine would be. It is posited that one of the advantages will be allowing users to focus better on the task they wish to perform, rather than on the interface, and thus abstract themselves into the experience with higher degree of sense of presence.
            
        %\item \textbf{Focusing on user culture contributes to Richness and Depth of interaction}\\
    \end{itemize}


\section{Thesis Structure} \label{sec:intro_structure}
    %Para além da introdução, esta dissertação contém mais x capítulos. No capítulo~\ref{chap:sota}, é descrito o estado da arte e são apresentados trabalhos relacionados.
    For the remainder of this document each chapter will focus on the background of each of three separate topics, their state of the art and findings that may be relevant.\\
    Chapter \ref{chap:back} will first focus on Culture in HCI and gestures from a cultural perspective, starting with an explanation of the Shamanic Interface proposal, its name. It will shortly reiterate some background knowledge required for the implementaton about Gesture Detection that was relevant to the research tools built for the prior dissertation.\\
    Chapter \ref{chap:develop} will cover the planning and development of the required systems and of the user trials. The progression through the chapter will fllow roughly after that of the development's work, starting with the reconstruction of the shamanic interface, going through the design elements present in the game, and ending with an overview of the trials, and their goals on each of its phases.\\
    Chapter \ref{chap:results} covers the results and analyisis obtained from the trials. The results will be covered following a similar order to the phases of the user trials, running from pre-test to post-test. Once that's done, the conclusions will summarize how the results fit into the goals posed above \ref{sec:intro_goals} by this work, and then concludes with some closing words and suggestions for future work.\\
    
