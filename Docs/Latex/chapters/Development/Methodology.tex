\chapter{Development} \label{chap:develop}
%\section*{} \label{sec:...}
%\begin{dummied}
%\end{dummied}
    As previously stated, this work is a continuation on the thesis by Tiago Susano Pinto\cite{pinto2015}. The goal of their work was the creation of a research tool that utilizes the shamanic interface concept, so empirical research can be performed with it.\\ 
    The developed software materializes the shamanic interface concept through its cultural layer, which is responsible for generating classifiers. These are through which the application will interchange gesture and command information. The cultural layer achieves this by storing relations between sequences of gestures, culture and meaning, and linking these to gesture models based on chosen culture. The models themselves are built following a Hidden Markov Machine approach, which in turn is available through the Accord.NET Framework for supervised learning. Models were recorded from hundreds of example gestures. As for the playable environment itself, it was built in Unity and it is controlled using a Leap Motion controller exclusively, which detects hand gestures performed by the user.\\
    So, in summation, the work involved developing three primary independent necessary components: The Shamanic Interface itself, which is implemented as library. A gesture recorder which is used to collect hand gesture data. And finally, a game volunteers are supposed to complete during their trials.\\
    During development of this work, all these components have been changed, either in small amounts for compatibility or feature addition, or as a complete rewrite and redesign. Additionally, completely new work was performed for the purposes of the user trials.


\section{Methodology} \label{sec:develop_methodology}
    The primary focus and methodology used in this work centres around a user trials in a controlled environment with a piece of immersive interactive content featuring hand movements. The main research goal is answering a number of questions regarding the approach and usage of a Shamanic Interface as a core building block of such kind of interactive content. These questions have not been previously answered due to the lack of prior research, and thus the intent is to formulate conclusive or precursory insight towards the realization of more complex implementations or systems for further research or prototype use.\\
    Briefly, the proposed increased benefits of enabling culturally enriched meaning through gestures by use of a Shamanic Interface, and the goals of this research, are the following: 
    \begin{itemize}
        \item Learning Speed and Capacity.
        \item Richness and Depth of interaction
        \item Retention and Memorization of commands and concepts.
    \end{itemize}
    It’s obvious that a diversity of domains is being looked at, such as psychology, culture and User Experience. Additionally, given that some or all of these benefits are already well-documented perks of Natural User Interfaces, which may cause trouble in evaluating the exact impact that is specific to the SI. Furthermore, a useful indicator of positive effect from one aspect of a user may be correlated with another n such a way that dissociating the two will lose the compounded benefit, which is a large concern in UX, such as visual and audible coherence.\\
    It’s important to find a method by which to distinguish the baseline benefit of a novel and immersive experience, and those specific to the application of the Shamanic Interface. Thus, it was reasoned that the user trials would be divided upon two groups. The first group would perform a culturally enriched experience similar to the previous works’ game, grounded on completing tasks in an openly explorable virtual space.  The second group would have to perform the exact same experience, however, the aspects that made the first culturally enriched have to be removed, without this affecting other variables, such as for example, the choice of gestures would still need to make a semblance of logical conveyance for the intended meaning, just not one that would be the person’s choice within the culture.\\
    There’re more concerns and even risks involved with making assumptions involved with the trials. One such assumption was with that of the designation of a Cultural Experience, and its broad meaning in literature review. To tackle this, it was necessary to include segments of interviews with the users at different timings of the trial dedicated to the problem. A semi-structured pre-test interview with exploratory intents, and a structured post-test interview handled step-by-step by the researcher following a protocol. Including these surveys, lengthens the total time required to complete a trial, which added to the expectation of an instructional step towards teaching the game to the volunteer, leads to the potential of apathetic exhaustion as a factor. \\
    Additionally, and separate from those issues, one of the required questions relates to the measure of memorability of the experience. This prompts the demand of a follow-up experience and, or, interview that that all volunteers partake in after a given delay of at least two weeks. This double phased trial allows determining the long-term impact given to the users, as well as verifying or inquiring about other behavioural indicators that may prove relevant.\\
    Thusly, changes to the Shamanic Interface and the development of the game were made with the early interest in mind with that a short, repeatable experience with swappable commands would be required. Before coming back and describing the specifics of the trials, this document will first go through the changes that the technical components of the implementation. So many of these changes may incur reasoning that relates to usability or the UX aspects of the application.