\section{Gesture Detection Methodologies} \label{sec:ges_}
    Gesture Recognition and Gesture Classification are two distinct aspects of the final application. Classification pertains to the attribution of meaning to a detected Gesture, and is therefore handled by the cultural layer, as each gesture may have different interpretation dependent on the culture chosen. Recognition is the prior step and is involved in how the system identifies meaningful gestures from among all gestures performed by the human users, in real time.\\
    There are a couple of approaches to handle recognition, between mathematical models and soft computing. Before tackling those, first it’d be important to quickly list the types of gestures that can be recognized and how they’re processed. These depend on the supporting instruments, which target different portions of the human body, obtain different sensor stimuli: Electric, Optic, Acoustic, Magnetic and Mechanic. These devices include: Gloves, Body Suits, Optical Trackers among others. Furthermore, vision-based techniques are incredibly varied and have several factors differentiating among themselves, by which broad fields of research and business are formed. The basic structure of a Gesture Recognition controller’s analysis involves two main tasks: Segmentation and Feature Extraction, features which are forwarded into a recognition module to build models. Segmentation is the extraction of the limbs of interest from background and determination of its location, while feature extraction is the determination of valuable data and cues among the segments. Not all input methods require the latter step, for example, magnetic sensors.\\
    The Recognition module follows up with different approaches: \emph{Hidden Markov Models} are a process governed by an underlying Markov chain with a finite number of states, and a random set of functions, each associated with each state. The transitions between states are based on probabilities. After each of a discrete amount of time, the system will be in one state and will observe a new symbol that feeds into the functions which will either yield a new state for the system, or output a recognized gesture for the followed chain. The chain involved a lot of mathematical modelling, producing a lot of deterministic integration, however, it can be seen as merely a sequence of observations and states, and thus it received the term “Hidden”. \emph{Particle Filtering} or alternatively, Sequential Monte Carlo, are approximations to simulation-based methods. Works by representing probabilities of noisy and partial samples, building a predictive model for the likelihood of following states. The benefit of these over grid-based filters such as conventional Markov models, is these do end up modelling uncertainty. \emph{The Finite State Machine} approach, by which a gesture is ordered by a sequence of states that vary in space and time. Each state is a datapoint of trajectory data, and the gesture is divided by each substantial change in trajectory data, sampled in a 2D space. Addition of gestures is achieved by constructing new FSM models and each gesture is matched to all the deterministic FSM’s. This does mean that adapting the system to more gestures requires incremental computational power, as well as adding winning criteria between gestures to choose the most likely when multiple are matched. \emph{Soft Computing}, which is a number of techniques which involve computational intelligence and computer learning with high degree of tolerance for imprecisions and uncertainty. The system can be trained even while in use and adapts to users. Methods include fuzzy logic, genetic algorithms, artificial neural networks. These, however, require a large amount of data and iterations to find adequate robustness.
